% !TEX TS-program = xelatex
% !TEX encoding = UTF-8 Unicode
% !Mode:: "TeX:UTF-8"

\documentclass{resume}
\usepackage{graphicx}
\usepackage{tabu}
\usepackage{multirow}
\usepackage{progressbar}
\usepackage{zh_CN-Adobefonts_external} % Simplified Chinese Support using external fonts (./fonts/zh_CN-Adobe/)
% \usepackage{NotoSansSC_external}
% \usepackage{NotoSerifCJKsc_external}
% \usepackage{zh_CN-Adobefonts_internal} % Simplified Chinese Support using system fonts
\usepackage{linespacing_fix} % disable extra space before next section
\usepackage{cite}
\usepackage{fontspec}

\usepackage{hyperref}
\hypersetup{
  colorlinks=true,
  linkcolor=blue,
  filecolor=blue,
  urlcolor=blue,
  citecolor=cyan,
}

\begin{document}
\pagenumbering{gobble} % suppress displaying page number

% \Large{
%   \begin{tabu}{ c l l }
%     & \scshape{Xueyao Zhang(张雪遥)} &  \\
%     & \email{xueyao\_98@foxmail.com} & \github[~~github.com/RMSnow]{https://github.com/RMSnow} \\
%     & \faHome{\href{https://www.zhangxueyao.com}{~~zhangxueyao.com}}
%     & \faGraduationCap {\href{https://scholar.google.com/citations?user=lf1udBcAAAAJ&hl=en}{~~Google Scholar}} 
%   \end{tabu}
% }

\name{Xueyao Zhang 张雪遥}

\basicInfo{
  % \faEnvelope{~xueyao\_98@foxmail.com} \textperiodcentered\
  \faEnvelope{~xueyaozhang@link.cuhk.edu.cn} \textperiodcentered\
  \faDesktop{\href{https://www.zhangxueyao.com}{~zhangxueyao.com}} \textperiodcentered\
  \faGraduationCap {\href{https://scholar.google.com/citations?user=lf1udBcAAAAJ&hl=en}{~Google Scholar}} }

\section{Education}
\datedsubsection{\textbf{School of Data Science, The Chinese University of Hong Kong, Shenzhen}}{Sept. 2022 -- }
{
  \small
  \begin{itemize}
    \item Ph.D. student in Data Science
    \item Areas of Study: Audio Generation, AI Music
    \item Advisor: Professor
          \href{https://scholar.google.com/citations?user=K6zhweAAAAAJ&hl=en}{Zhizheng
            Wu}
  \end{itemize}
}

\datedsubsection{\textbf{Institute of Computing Technology, Chinese Academy of Sciences}}{Sept. 2019 -- Jun. 2022}
{
  \small
  % 前瞻实验室,研究方向:虚假新闻检测、事实核查,导师:\href{http://www.ict.cas.cn/sourcedb_2018_ict_cas/cn/jssrck/201011/t20101123_3028158.html}{曹娟}研究员;GPA:3.79 / 4.00.
  \begin{itemize}
    \item Master in Computer Application Technology (being recommended for admission)
    \item Areas of Study: Fake News Detection, Fact Checking
    \item Advisor: Professor
          \href{https://scholar.google.com/citations?user=fSBdNg0AAAAJ&hl=zh-CN}{Juan
            Cao}
    \item GPA: 3.79 / 4.00, Math-related courses: 91 / 100
  \end{itemize}
}

\datedsubsection{\textbf{School of Computer Science, Wuhan University}}{Sept. 2015 -- Jun. 2019}
{
  \small
  % GPA:3.84 / 4.00,专业排名:4 / 246,英语六级:522.
  \begin{itemize}
    \item B.Eng. in Software Engineering
    \item GPA: 3.83 / 4.00, Ranking: 4 / 246 (Top 1.6\%)
    \item Math-related courses: Advanced Mathematics (95), Linear Algebra (94), Discrete
          Mathematics (93)
  \end{itemize}
}

\section{Representative Works}

\begin{enumerate}\itemsep 0.5em
  \item \textbf{Xueyao Zhang}*, Liumeng Xue*, Yuancheng Wang*, Yicheng Gu*, et al. Amphion: An Open-Source Audio, Music and Speech Generation Toolkit. \href{https://arxiv.org/pdf/2312.09911.pdf}{[Technical Report]} \href{https://github.com/open-mmlab/Amphion}{[GitHub]} \href{https://huggingface.co/amphion}{[HuggingFace]}
  \item \textbf{Xueyao Zhang}, et al. Leveraging Content-based Features from Multiple Acoustic Models for Singing Voice Conversion. \textit{Machine Learning for Audio Workshop at NeuIPS 2023.} \href{https://arxiv.org/pdf/2310.11160.pdf}{[Paper]} \href{https://github.com/open-mmlab/Amphion/tree/main/egs/svc/MultipleContentsSVC}{[Code]} \href{https://www.zhangxueyao.com/data/MultipleContentsSVC/index.html}{[Demo]}
  \item \textbf{Xueyao Zhang}, et al. Structure-Enhanced Pop Music Generation via Harmony-Aware Learning. \textit{Proceedings of the 30th ACM International Conference on Multimedia (ACM MM 2022).} \href{https://dl.acm.org/doi/10.1145/3503161.3548084}{[Paper]} \href{https://github.com/RMSnow/HAT}{[Code]} \href{https://www.zhangxueyao.com/data/HAT/demo.html}{[Demo]}
  \item \textbf{Xueyao Zhang}, et al. Mining Dual Emotion for Fake News Detection. \textit{Proceedings of the 30th Web Conference (WWW 2021).} \href{https://dl.acm.org/doi/pdf/10.1145/3442381.3450004}{[Paper]} \href{https://github.com/RMSnow/WWW2021}{[Code]} \href{https://www.bilibili.com/video/BV13o4y1m7c3}{[Video]}
  \item Qiang Sheng*, \textbf{Xueyao Zhang}*, et al. Integrating Pattern- and
        Fact-based Fake News Detection via Model Preference Learning.
        \textit{Proceedings of the 30th ACM International Conference on Information and
          Knowledge Management (CIKM 2021).} \href{https://dl.acm.org/doi/10.1145/3459637.3482440}{[Paper]}
          \href{https://github.com/ICTMCG/Pref-FEND}{[Code]}
          \href{https://zhuanlan.zhihu.com/p/414464291}{[Blog]}
\end{enumerate}
*: Equal Contribution.

\section{Internships}
\datedsubsection{\faWechat{} \textbf{WeChat, Tencent}}{Apr. 2021 -- Feb. 2022,\quad June 2023 --}
{\small \role{Research Intern, Pattern Recognition Center of Wechat AI, Beijing, China
  }{}
}
% \small
\begin{itemize}
  \item My research topic is \textit{AI + Music}. Besides, I supply the musical knowledge support for our
        team.
  \item In May 2021, I gave a talk on music,
        \href{https://www.zhangxueyao.com/data/wcpr-pop-music.pdf}{\underline{How to
            create a pop song? \textit{(一首流行歌是如何创作的)}}}
  \item In Jan 2022, I participated to release the theme song of WeChat Open Class 2022
        as a co-producer, \href{https://y.qq.com/n/ryqq/songDetail/000xeNJ53orPG2}{Ru
          Wei \textit{(入微)}}, which was composed by AI and sung by humans.
  \item In June 2023, I was admitted by Tencent Rhino-Bird Talent Program \textit{腾讯犀牛鸟精英人才计划}
        (Top 50+ of China), supervised by
        \href{https://scholar.google.com/citations?user=vH9YLsAAAAAJ&hl=en}{Jinchao
          Zhang}.
\end{itemize}

\section{Services}
\datedsubsection{\textbf{Reviewer}}{}
\begin{itemize}
  \item Conferences: ACL Rolling Review, ACM CSCW 2021, ACM MM 2023, EMNLP 2021, ICMC 2023, ICASSP 2023
        \href{https://www.zhangxueyao.com/data/presentations/2023_icassp_reviewer.pdf}{(valuable
          reviewer)} / 2024, NCMMSC 2023
  \item Journals: EURASIP Journal on Audio, Speech, and Music Processing, IEEE
        Transactions on Audio, Speech and Language Processing (TASLP), IEEE
        Transactions on Computational Social Systems (TCSS), Information Processing and
        Management (IP\&M), Journal of Chinese Information Processing \textit{(中文信息学报)}
\end{itemize}
\datedsubsection{\textbf{Teaching Assistant}}{}
\begin{itemize}
  \item Fall 2017, Object-Oriented Programming (JAVA), Wuhan University
  \item Fall 2022, CSC3100 Data Structures, CUHK-Shenzhen
  \item Spring 2023, CSC3160/MDS6002
        \href{https://drwuz.com/CSC3160/index.html}{Fundamentals of Speech and Language
          Processing}, CUHK-Shenzhen,
        \href{https://www.zhangxueyao.com/data/presentations/20230521_ta_award.pdf}{Best
          Teaching Assistant}.
  \item Fall 2023, CSC4130 \href{https://stevenhan1991.github.io/CSC4130/index.html}{Introduction to Human-Computer Interaction}, CUHK-Shenzhen
\end{itemize}

\section{Honors and Awards}
\begin{enumerate}
  \item Tencent Rhino-Bird Talent Program \textit{腾讯犀牛鸟精英人才计划} (Top 50+ of China)
  \item National Graduate Scholarship, Ministry of Education of China \textit{研究生国家奖学金}
        (Top 0.2\%, 2021)
  \item National Undergraduate Scholarship, Ministry of Education of China
        \textit{本科生国家奖学金} (Top 0.2\%, 2016)
  \item Third place at Campus Singer Competition, University of Chinese Academy of
        Sciences \textit{中国科学院大学校园歌手大赛季军} (Top 3, 2020)
  \item Outstanding Graduate, University of Chinese Academy of Sciences
        \textit{中国科学院大学优秀毕业生} (Top 5\%, 2022)
  \item Outstanding Graduate, Beijing Municipal Education Commission \textit{北京市优秀毕业生}
        (Top 5\%, 2022)
  \item Excellent Bachelor Thesis, Wuhan University \textit{武汉大学优秀毕业论文} (Top 5\%, 2019)
  \item First price in Chinese High School Mathematics League \textit{全国高中数学联赛一等奖} (Top
        50 in Henan Province, 2014)
  \item Merit Student, University of Chinese Academy of Sciences \textit{中国科学院大学三好学生}
        (2020; 2021; 2022)
  \item Merit Student and Outstanding Student Leaders, Wuhan University
        \textit{武汉大学三好学生、优秀班干部} (2016; 2017; 2018)
\end{enumerate}

\section{Musical Abilities}
\begin{itemize}
  \item Familiar with the basic music theory.
  \item Familiar with the common musical genres.
  \item Proficient in playing pop keyboard.
  \item Proficient in pop singing and familiar with Bel canto.
\end{itemize}

\section{Programming Skills}
\begin{itemize}
  \item Programming Languages: proficient in Python, familiar with Java, and know about
        C++.
  \item Deep Learning Frameworks: proficient in PyTorch and Keras.
\end{itemize}

\end{document}