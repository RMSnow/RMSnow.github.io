% !TEX TS-program = xelatex
% !TEX encoding = UTF-8 Unicode
% !Mode:: "TeX:UTF-8"

\documentclass{resume}
\usepackage{graphicx}
\usepackage{tabu}
\usepackage{multirow}
\usepackage{progressbar}
\usepackage{zh_CN-Adobefonts_external} % Simplified Chinese Support using external fonts (./fonts/zh_CN-Adobe/)
% \usepackage{NotoSansSC_external}
% \usepackage{NotoSerifCJKsc_external}
% \usepackage{zh_CN-Adobefonts_internal} % Simplified Chinese Support using system fonts
\usepackage{linespacing_fix} % disable extra space before next section
\usepackage{cite}
\usepackage{fontspec}

\usepackage{hyperref}
\hypersetup{
    colorlinks=false,
    linkcolor=blue,
    filecolor=blue,      
    urlcolor=blue,
    citecolor=cyan,
}

\begin{document}
\pagenumbering{gobble} % suppress displaying page number

% \Large{
%   \begin{tabu}{ c l l }
%     & \scshape{Xueyao Zhang(张雪遥)} &  \\
%     & \email{xueyao\_98@foxmail.com} & \github[~~github.com/RMSnow]{https://github.com/RMSnow} \\
%     & \faHome{\href{https://www.zhangxueyao.com}{~~zhangxueyao.com}}
%     & \faGraduationCap {\href{https://scholar.google.com/citations?user=lf1udBcAAAAJ&hl=en}{~~Google Scholar}} 
%   \end{tabu}
% }

\name{Xueyao Zhang 张雪遥}

\basicInfo{
  % \faEnvelope{~xueyao\_98@foxmail.com} \textperiodcentered\
  \faEnvelope{~xueyaozhang@link.cuhk.edu.cn} \textperiodcentered\ 
  \faDesktop{\href{https://www.zhangxueyao.com}{~zhangxueyao.com}} \textperiodcentered\ 
  \faGraduationCap {\href{https://scholar.google.com/citations?user=lf1udBcAAAAJ&hl=en}{~Google Scholar}} }


\section{Education}
\datedsubsection{\textbf{School of Data Science, The Chinese University of Hong Kong, Shenzhen}}{Sept. 2022 -- }
{
  \small 
\begin{itemize}
  \item Ph.D. student in Data Science
  \item Areas of Study: Automatic Music Creation, especially on Singing Voice Synthesis / Conversion
  \item Advisor: Professor \href{https://scholar.google.com/citations?user=K6zhweAAAAAJ&hl=en}{Zhizheng Wu} 
\end{itemize}
}

\datedsubsection{\textbf{Institute of Computing Technology, Chinese Academy of Sciences}}{Sept. 2019 -- Jun. 2022}
{
  \small 
  % 前瞻实验室,研究方向:虚假新闻检测、事实核查,导师:\href{http://www.ict.cas.cn/sourcedb_2018_ict_cas/cn/jssrck/201011/t20101123_3028158.html}{曹娟}研究员;GPA:3.79 / 4.00.
\begin{itemize}
  \item Master in Computer Application Technology (being recommended for admission)
  \item Areas of Study: Computational Social Science, especially on Fake News Detection
  \item Advisor: Professor \href{https://scholar.google.com/citations?user=fSBdNg0AAAAJ&hl=zh-CN}{Juan Cao}
  \item GPA: 3.79 / 4.00, Math-related courses: 91 / 100
\end{itemize}
}

\datedsubsection{\textbf{School of Computer Science, Wuhan University}}{Sept. 2015 -- Jun. 2019}
{
  \small 
  % GPA:3.84 / 4.00,专业排名:4 / 246,英语六级:522.
\begin{itemize}
  \item B.Eng. in Software Engineering
  \item GPA: 3.83 / 4.00, Ranking: 4 / 246 (Top 1.6\%)
  \item Math-related courses: Advanced Mathematics (95), Linear Algebra (94), Discrete Mathematics (93)
\end{itemize}
}


\section{Publications}

\begin{enumerate}\itemsep 0.5em
  \item \textbf{Xueyao Zhang}, Jinchao Zhang, Yao Qiu, Li Wang, Jie Zhou.\\Structure-Enhanced Pop Music Generation via Harmony-Aware Learning. \textit{Proceedings of the 30th ACM International Conference on Multimedia (ACM MM 2022).}
  \item \textbf{Xueyao Zhang}, Juan Cao, Xirong Li, Qiang Sheng, Lei Zhong, Kai Shu.\\Mining Dual Emotion for Fake News Detection. \textit{Proceedings of the 30th Web Conference (WWW 2021).}
  \item Qiang Sheng*, \textbf{Xueyao Zhang}*, Juan Cao, Lei Zhong.  (*: Equal Contribution)\\Integrating Pattern- and Fact-based Fake News Detection via Model Preference Learning. \textit{Proceedings of the 30th ACM International Conference on Information and Knowledge Management (CIKM 2021).}
  \item Qiang Sheng, Juan Cao, \textbf{Xueyao Zhang}, Xirong Li, Lei Zhong.\\Article Reranking by Memory-enhanced Key Sentence Matching for Detecting Previously Fact-checked Claims. \textit{Proceedings of the Joint Conference of the 59th Annual Meeting of the Association for Computational Linguistics and the 11th International Joint Conference on Natural Language Processing (ACL-IJCNLP 2021) }
  \item Qiang Sheng, Juan Cao, \textbf{Xueyao Zhang}, Rundong Li, Danding Wang, Yongchun Zhu.\\Zoom Out and Observe: News Environment Perception for Fake News Detection. \textit{Proceedings of the Joint Conference of the 60th Annual Meeting of the Association for Computational Linguistics (ACL 2022).}
\end{enumerate}


\section{Internships}
\datedsubsection{\faWechat{} \textbf{WeChat, Tencent}}{Apr. 2021 -- Feb. 2022}
{\small \role{Research Intern, Pattern Recognition Center of Wechat AI, Beijing, China
}{}
}
% \small
\begin{itemize}
  \item My research topic is \textit{symbolic music generation}. Besides, I supply the technical support of musicology knowledge (including music theory) for our team. 
  \item In May 2021, I gave a talk on music, \href{https://www.zhangxueyao.com/data/wcpr-pop-music.pdf}{\underline{How to create a pop song? \textit{(一首流行歌是如何创作的)}}}
  \item In Jan 2022, I participated to release the theme song of WeChat Open Class 2022 as a co-producer, \href{https://y.qq.com/n/ryqq/songDetail/000xeNJ53orPG2}{Ru Wei \textit{(入微)}}, which was composed by AI and sung by humans.
\end{itemize}

\section{Services}
\datedsubsection{\textbf{Reviewer}}{}
\begin{itemize}
  \item Conferences: ACL Rolling Review, ACM MM 2022, EMNLP 2021, ACM CSCW 2021
  \item Journals: Information Processing and Management (IP\&M), Journal of Chinese Information Processing \textit{(中文信息学报)}
\end{itemize}
\datedsubsection{\textbf{Teaching Assistant}}{}
\begin{itemize}
  \item 2017 Fall, Object-Oriented Programming (JAVA), Wuhan University
\end{itemize}

\section{Honors and Awards}
\begin{enumerate}
  \item National Graduate Scholarship, Ministry of Education of China \textit{研究生国家奖学金} (Top 0.2\%, 2021)
  \item National Undergraduate Scholarship, Ministry of Education of China \textit{本科生国家奖学金} (Top 0.2\%, 2016)
  \item Third place at Campus Singer Competition, University of Chinese Academy of Sciences \textit{中国科学院大学校园歌手大赛季军} (Top 3, 2020)
  \item Outstanding Graduate, University of Chinese Academy of Sciences \textit{中国科学院大学优秀毕业生} (Top 5\%, 2022)
  \item Outstanding Graduate, Beijing Municipal Education Commission \textit{北京市优秀毕业生} (Top 5\%, 2022)
  \item Excellent Bachelor Thesis, Wuhan University \textit{武汉大学优秀毕业论文} (Top 5\%, 2019)
  \item First price in Chinese High School Mathematics League \textit{全国高中数学联赛一等奖} (Top 50 in Henan Province, 2014)
  \item Merit Student, University of Chinese Academy of Sciences \textit{中国科学院大学三好学生} (2020; 2021; 2022)
  \item Merit Student and Outstanding Student Leaders, Wuhan University \textit{武汉大学三好学生、优秀班干部} (2016; 2017; 2018)
\end{enumerate}

\section{Musical Abilities}
\begin{itemize}
  \item Familiar with the basic music theory.
  \item Familiar with the common musical genres.
  \item Proficient in playing pop keyboard.
  \item Proficient in pop singing and familiar with Bel canto.
\end{itemize}

\section{Programming Skills}
\begin{itemize}
  \item Programming Languages: proficient in Python, familiar with Java, and know about C++.
  \item Deep Learning Frameworks: proficient in PyTorch and Keras.
\end{itemize}

\end{document}
