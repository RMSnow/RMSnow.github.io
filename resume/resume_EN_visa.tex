% !TEX TS-program = xelatex
% !TEX encoding = UTF-8 Unicode
% !Mode:: "TeX:UTF-8"

\documentclass{resume}
\usepackage{graphicx}
\usepackage{tabu}
\usepackage{multirow}
\usepackage{progressbar}
\usepackage{zh_CN-Adobefonts_external} % Simplified Chinese Support using external fonts (./fonts/zh_CN-Adobe/)
% \usepackage{NotoSansSC_external}
% \usepackage{NotoSerifCJKsc_external}
% \usepackage{zh_CN-Adobefonts_internal} % Simplified Chinese Support using system fonts
\usepackage{linespacing_fix} % disable extra space before next section
\usepackage{cite}
\usepackage{fontspec}

\usepackage{hyperref}
\hypersetup{
  colorlinks=true,
  linkcolor=blue,
  filecolor=blue,
  urlcolor=blue,
  citecolor=cyan,
}

\begin{document}
\pagenumbering{gobble} % suppress displaying page number

% \Large{
%   \begin{tabu}{ c l l }
%     & \scshape{Xueyao Zhang(张雪遥)} &  \\
%     & \email{xueyao\_98@foxmail.com} & \github[~~github.com/RMSnow]{https://github.com/RMSnow} \\
%     & \faHome{\href{https://www.zhangxueyao.com}{~~zhangxueyao.com}}
%     & \faGraduationCap {\href{https://scholar.google.com/citations?user=lf1udBcAAAAJ&hl=en}{~~Google Scholar}} 
%   \end{tabu}
% }

\name{Xueyao Zhang 张雪遥}

\basicInfo{
  % \faEnvelope{~xueyao\_98@foxmail.com} \textperiodcentered\
  \faEnvelope{~xueyaozhang@link.cuhk.edu.cn} \textperiodcentered\
  \faDesktop{\href{https://www.zhangxueyao.com}{~zhangxueyao.com}} \textperiodcentered\
  \faGraduationCap {\href{https://scholar.google.com/citations?user=lf1udBcAAAAJ&hl=en}{~Google Scholar}} }

\section{Publications}

\begin{enumerate}\itemsep 0.5em
  \item \textbf{Xueyao Zhang}, et al. Structure-Enhanced Pop Music Generation via Harmony-Aware Learning. \textit{Proceedings of the 30th ACM International Conference on Multimedia (ACM MM 2022).} \href{https://dl.acm.org/doi/10.1145/3503161.3548084}{[Paper]} \href{https://github.com/RMSnow/HAT}{[Code]} \href{https://www.zhangxueyao.com/data/HAT/demo.html}{[Demo]}
  \item \textbf{Xueyao Zhang}*, Liumeng Xue*, Yuancheng Wang*, Yicheng Gu*, et al. Amphion: An Open-Source Audio, Music and Speech Generation Toolkit. \href{https://arxiv.org/pdf/2312.09911.pdf}{[Technical Report]} \href{https://github.com/open-mmlab/Amphion}{[GitHub]} \href{https://huggingface.co/amphion}{[HuggingFace]}
  \item \textbf{Xueyao Zhang}, et al. Leveraging Content-based Features from Multiple Acoustic Models for Singing Voice Conversion. \textit{Audio Generation Workshop at NeuIPS 2023.} \href{https://arxiv.org/pdf/2310.11160.pdf}{[Paper]} \href{https://github.com/open-mmlab/Amphion/tree/main/egs/svc/MultipleContentsSVC}{[Code]} \href{https://www.zhangxueyao.com/data/MultipleContentsSVC/index.html}{[Demo]}
\end{enumerate}
*: Equal Contribution.

\section{Internships}
\datedsubsection{\faWechat{} \textbf{WeChat, Tencent}}{Apr. 2021 -- Feb. 2022,\quad June 2023 --}
{\small \role{Music Generation Research Intern, Beijing, China
  }{}
}
% \small
\begin{itemize}
  \item My research topic is \textit{Music Generation}. Besides, I provide the musical knowledge support for our
        team.
  \item In May 2021, I gave a talk on music,
        \href{https://www.zhangxueyao.com/data/wcpr-pop-music.pdf}{\underline{How to
            create a pop song? \textit{(一首流行歌是如何创作的)}}}
  \item In Jan 2022, I participated to release the theme song of WeChat Open Class 2022
        as a co-producer, \href{https://y.qq.com/n/ryqq/songDetail/000xeNJ53orPG2}{Ru
          Wei \textit{(入微)}}, which was composed by computer and sung by humans.
\end{itemize}

\section{Education}
% \datedsubsection{\textbf{The Chinese University of Hong Kong, Shenzhen}}{Sept. 2022 -- }
{
  \small
  \textbf{The Chinese University of Hong Kong, Shenzhen} (Sept. 2022 -- Jun. 2026)
  \begin{itemize}
    \item Ph.D. student, Areas of Study: Music Generation
    \item Advisor: Professor
          \href{https://scholar.google.com/citations?user=K6zhweAAAAAJ&hl=en}{Zhizheng
            Wu}
  \end{itemize}
  \textbf{University of Chinese Academy of Sciences} (Sept. 2019 -- Jun. 2022)
  \small
  \begin{itemize}
    \item Master, Areas of Study: Computational Social Science
    \item Advisor: Professor
          \href{https://scholar.google.com/citations?user=fSBdNg0AAAAJ&hl=zh-CN}{Juan
            Cao}
  \end{itemize}
  \textbf{Wuhan University} (Sept. 2015 -- Jun. 2019)
  \begin{itemize}
    \item B.Eng. in Software Engineering
    \item GPA: 3.83 / 4.00, Ranking: 4 / 246 (Top 1.6\%)
    \item Math-related courses: Advanced Mathematics (95), Linear Algebra (94), Discrete
          Mathematics (93)
  \end{itemize}
}

\section{Honors and Awards}
\begin{enumerate}
  % \item Tencent Rhino-Bird Talent Program \textit{腾讯犀牛鸟精英人才计划} (Top 50+ of China)
  \item Third place at Campus Singer Competition, University of Chinese Academy of
        Sciences \textit{中国科学院大学校园歌手大赛季军} (Top 3, 2020)
  \item Outstanding Graduate, University of Chinese Academy of Sciences
        \textit{中国科学院大学优秀毕业生} (Top 5\%, 2022)
  \item Excellent Bachelor Thesis, Wuhan University \textit{武汉大学优秀毕业论文} (Top 5\%, 2019)
  \item Merit Student, University of Chinese Academy of Sciences \textit{中国科学院大学三好学生}
        (2020; 2021; 2022)
  \item Merit Student and Outstanding Student Leaders, Wuhan University
        \textit{武汉大学三好学生、优秀班干部} (2016; 2017; 2018)
\end{enumerate}

\section{Musical Abilities}
\begin{itemize}
  \item Familiar with the basic music theory.
  \item Familiar with the common musical genres.
  \item Proficient in playing pop keyboard.
  \item Proficient in pop singing and familiar with Bel canto.
\end{itemize}

\end{document}