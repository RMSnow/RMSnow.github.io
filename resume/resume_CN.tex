% !TEX TS-program = xelatex
% !TEX encoding = UTF-8 Unicode
% !Mode:: "TeX:UTF-8"

\documentclass{resume}
\usepackage{graphicx}
\usepackage{tabu}
\usepackage{multirow}
\usepackage{progressbar}
\usepackage{zh_CN-Adobefonts_external} % Simplified Chinese Support using external fonts (./fonts/zh_CN-Adobe/)
% \usepackage{NotoSansSC_external}
% \usepackage{NotoSerifCJKsc_external}
% \usepackage{zh_CN-Adobefonts_internal} % Simplified Chinese Support using system fonts
\usepackage{linespacing_fix} % disable extra space before next section
\usepackage{cite}
\usepackage{fontspec}

\usepackage{hyperref}
\hypersetup{
    colorlinks=true,
    linkcolor=blue,
    filecolor=blue,      
    urlcolor=blue,
    citecolor=cyan,
}

\begin{document}
\pagenumbering{gobble} % suppress displaying page number

% \Large{
%   \begin{tabu}{ c l l }
%     & \scshape{Xueyao Zhang(张雪遥)} &  \\
%     & \email{xueyao\_98@foxmail.com} & \github[~~github.com/RMSnow]{https://github.com/RMSnow} \\
%     & \faHome{\href{https://www.zhangxueyao.com}{~~zhangxueyao.com}}
%     & \faGraduationCap {\href{https://scholar.google.com/citations?user=lf1udBcAAAAJ&hl=en}{~~Google Scholar}} 
%   \end{tabu}
% }

\name{张雪遥}

\basicInfo{
  % \faEnvelope{~xueyao\_98@foxmail.com} \textperiodcentered\
  \faEnvelope{~xueyaozhang@link.cuhk.edu.cn} \textperiodcentered\ 
  \faDesktop{\href{https://www.zhangxueyao.com}{~zhangxueyao.com}} \textperiodcentered\ 
  \faGraduationCap {\href{https://scholar.google.com/citations?user=lf1udBcAAAAJ&hl=en}{~Google Scholar}} }


\section{教育背景}
\datedsubsection{\textbf{香港中文大学(深圳)- 数据科学学院}}{2022年9月 - }
{
  \small 
\begin{itemize}
  \item 数据科学专业博士生,导师:\href{https://scholar.google.com/citations?user=K6zhweAAAAAJ&hl=en}{武执正}教授 
  \item 研究领域:智能化音乐创作,尤其专注于歌声转换(Singing Voice Conversion)
  % \item 导师:\href{https://scholar.google.com/citations?user=K6zhweAAAAAJ&hl=en}{武执正}教授 
\end{itemize}
}

\datedsubsection{\textbf{中国科学院计算技术研究所}}{2019年9月 - 2022年6月}
{
  \small 
\begin{itemize}
  \item 计算机应用技术专业硕士(保研),导师:\href{https://scholar.google.com/citations?user=fSBdNg0AAAAJ&hl=zh-CN}{曹娟}研究员
  \item 研究领域:虚假新闻检测、事实核查
  \item GPA:3.79 / 4.00, 数学类课程:91 / 100
\end{itemize}
}

\datedsubsection{\textbf{武汉大学 - 计算机学院}}{2015年9月 -  2019年6月}
{
  \small 
\begin{itemize}
  \item 软件工程专业,GPA:3.83 / 4.00, 专业排名: 4 / 246 (Top 1.6\%)
  \item 数学类课程: 高等数学(95), 线性代数(94),离散数学(93)
\end{itemize}
}


\section{发表论文}

\begin{enumerate}\itemsep 0.5em
  \item \textbf{Xueyao Zhang}, Jinchao Zhang, Yao Qiu, Li Wang, Jie Zhou. \href{https://dl.acm.org/doi/10.1145/3503161.3548084}{[PDF]} \href{https://github.com/RMSnow/HAT}{[Code]} \href{https://www.zhangxueyao.com/data/HAT/slides.pdf}{[Slides]} \href{https://www.zhangxueyao.com/data/HAT/demo.html}{[Demo]}\\Structure-Enhanced Pop Music Generation via Harmony-Aware Learning. \textit{Proceedings of the 30th ACM International Conference on Multimedia (ACM MM 2022).}
  \item \textbf{Xueyao Zhang}, Juan Cao, Xirong Li, Qiang Sheng, Lei Zhong, Kai Shu. \href{https://dl.acm.org/doi/pdf/10.1145/3442381.3450004}{[PDF]} \href{https://github.com/RMSnow/WWW2021}{[Code]} \href{https://www.zhangxueyao.com/assets/www2021-dual-emotion-slides.pdf}{[Slides]} \href{https://www.bilibili.com/video/BV13o4y1m7c3}{[Video]}\\ Mining Dual Emotion for Fake News Detection. \textit{Proceedings of the 30th Web Conference (WWW 2021).}
  \item Qiang Sheng*, \textbf{Xueyao Zhang}*, Juan Cao, Lei Zhong.  (*: Equal Contribution) \href{https://dl.acm.org/doi/10.1145/3459637.3482440}{[PDF]} \href{https://github.com/ICTMCG/Pref-FEND}{[Code]} \href{https://zhuanlan.zhihu.com/p/414464291}{[Blog]}\\Integrating Pattern- and Fact-based Fake News Detection via Model Preference Learning. \textit{Proceedings of the 30th ACM International Conference on Information and Knowledge Management (CIKM 2021).}
  \item Qiang Sheng, Juan Cao, \textbf{Xueyao Zhang}, Xirong Li, Lei Zhong. \href{https://aclanthology.org/2021.acl-long.425.pdf}{[PDF]} \href{https://github.com/ICTMCG/MTM}{[Code]} \href{https://zhuanlan.zhihu.com/p/393615707}{[Blog]}\\Article Reranking by Memory-enhanced Key Sentence Matching for Detecting Previously Fact-checked Claims. \textit{Proceedings of the Joint Conference of the 59th Annual Meeting of the Association for Computational Linguistics and the 11th International Joint Conference on Natural Language Processing (ACL-IJCNLP 2021) }
  \item Qiang Sheng, Juan Cao, \textbf{Xueyao Zhang}, Rundong Li, Danding Wang, Yongchun Zhu. \href{https://aclanthology.org/2022.acl-long.311.pdf}{[PDF]} \href{https://github.com/ICTMCG/News-Environment-Perception}{[Code]} \href{https://mp.weixin.qq.com/s/aTFeuCYIpSoazeRi52jqew}{[Blog]}\\Zoom Out and Observe: News Environment Perception for Fake News Detection. \textit{Proceedings of the Joint Conference of the 60th Annual Meeting of the Association for Computational Linguistics (ACL 2022).}
\end{enumerate}

\section{发表专利}
\small{
  \begin{enumerate}\itemsep 0.5em
    \item CN113239685A.曹娟;\textbf{张雪遥};盛强;谢添;李锦涛.一种基于双重情感的舆情检测方法及系统
    \item CN113536760A . 曹娟;盛强;\textbf{张雪遥};钟雷;谢添 . 引述句和辟谣模式句引导的“谣言-辟谣文章”匹配方法及系统
    \item CN113849599A . 曹娟;盛强;\textbf{张雪遥};钟雷;谢添 . 基于模式信息和事实信息的联合虚假新闻检测方法
    \item CN202210214207.3(申请号). 曹娟;盛强;\textbf{张雪遥} . 基于新闻环境信息建模的虚假新闻检测方法
  \end{enumerate}
}


\section{实习经历}
\datedsubsection{\faWechat{} \textbf{腾讯 - 微信 - 模式识别中心}}{2021年4月 - 2022年2月}
{\small \role{音乐生成算法实习生}{}
}
% \small
\begin{itemize}
  \item 研究课题:符号化音乐生成;为团队提供音乐背景
  乐理知识等支持。
  \item 2021年5月,举办一次内部的音乐培训讲座:\href{https://www.zhangxueyao.com/data/wcpr-pop-music.pdf}{\underline{一首流行歌是如何创作的}}。
  \item 2022年1月,参与发布2022年微信公开课主题曲:\href{https://y.qq.com/n/ryqq/songDetail/000xeNJ53orPG2}{\underline{《入微》}}。
\end{itemize}

\section{服务工作}
\datedsubsection{\textbf{学术审稿}}{}
\begin{itemize}
  \item 会议:ICASSP 2023, ACL Rolling Review, ACM MM 2022, EMNLP 2021, ACM CSCW 2021
  \item 期刊: Information Processing and Management (IP\&M), EURASIP Journal on Audio, Speech, and Music Processing, 中文信息学报
\end{itemize}
\datedsubsection{\textbf{助教}}{}
\begin{itemize}
  \item 2017年秋季:面向对象的编程(Java),武汉大学
  \item 2022年秋季:CSC3100 数据结构,香港中文大学(深圳)
  \item 2023年春季:CSC3160/MDS6002 \href{https://drwuz.com/CSC3160/index.html}{语音与语言处理基础},香港中文大学(深圳)
\end{itemize}

\section{获奖情况}
\begin{enumerate}
  \item 研究生国家奖学金 (Top 0.2\%, 2021)
  \item 本科生国家奖学金 (Top 0.2\%, 2016)
  \item 中国科学院大学校园歌手大赛季军 (Top 3, 2020)
  \item 中国科学院大学优秀毕业生 (Top 5\%, 2022)
  \item 北京市优秀毕业生 (Top 5\%, 2022)
  \item 武汉大学优秀毕业论文 (Top 5\%, 2019)
  \item 全国高中数学联赛一等奖 (河南省Top 50, 2014)
  \item 中国科学院大学三好学生 (2020; 2021; 2022)
  \item 武汉大学三好学生、优秀班干部 (2016; 2017; 2018)
\end{enumerate}

\section{音乐能力}
\begin{itemize}
  \item 掌握基础的乐理知识,熟悉常见的音乐风格
  \item 熟悉流行音乐的键盘演奏,了解吉他演奏
  \item 爱好流行演唱,曾获国科大校园歌手大赛季军;拥有合唱团演唱、领唱经历,了解基础的美声知识
\end{itemize}

% \section{编程技能}
% \begin{itemize}
%   \item 编程语言:熟练使用Python,了解Java、C++
%   \item 深度学习框架:熟练使用PyTorch
% \end{itemize}

\end{document}
